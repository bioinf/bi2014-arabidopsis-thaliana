\documentclass[12pt,a4paper]{article}
\usepackage[utf8]{inputenc}
\usepackage{amsmath}
\usepackage{amsfonts}
\usepackage{amssymb}
\usepackage{graphicx}
\usepackage{datetime}
\usepackage{lipsum}
\usepackage{mathtools}

\usepackage[T2A,T1]{fontenc}
\usepackage[utf8]{inputenc}
\usepackage[russian]{babel}

\usepackage{titlesec}

\author{Arabidopsis team}
\title{Arabidopsis thaliana study report}

\begin{document}
	
	\titleformat{\section}[block]{\bf\centering}{}{}{}{}
	\section*{Отчёт по изучению Arabidopsis Thaliana}
	\begin{center}
	    Документ создан \today.
	    \vskip\medskipamount % or other desired dimension
	    \leaders\vrule width \textwidth\vskip0.5pt % or other desired thickness
	    \vskip\medskipamount % ditto
	    \nointerlineskip
	\end{center}
	
	\begin{itemize}
		\item \bf Выбор родственного организма \normalfont
	    
	    Используется бобовое растение люцерна. По филогенетическому дереву двудольных, семейство бобовых находится ближе к семейству Капустных (к которому относится арабидопсис), чем также секвенированное семейство Паслёновые.

		Краткую русскую информацию можно почитать здесь:\\
		$https://ru.wikipedia.org/wiki/Medicago_truncatula$
		
		Ссылка на скачивание генома:\\
		$http://www.jcvi.org/medicago/display.php?pageName=General\&section=Download$
	
		Полная карта каждой из 8 хромосом:\\
		$http://www.medicagohapmap.org/downloads/mt40$
	    
	    \item \bf Blastn \normalfont
	    
	    Необходимо произвести выравнивание всех доступных Expressed sequence tags (ESTs) организма Mendicago truncatula на геном Arabidopsis Thaliana при помощи Blastn, чтобы найти гомологичные последовательности в геноме организма Arabidopis Thaliana (скрипт $run\_blastn.sh$).
	    
	    ESTs доступны для скачивания по адресу:\\
	    $www.plantgdb.org/download/Download/PublicPlantSeq/Dump/M/\\
	    Medicago_truncatula/FASTA/
	    Medicago_truncatula.mRNA.EST.fasta.bz2$
	    
	    Выходной файл представляет собой $tab$-separated файл вида: 1-й и 2-й столбцы содержат начало и конец найденной гомологичной последовательности, а 3-й и 4-й - её $id$ и представление, соответственно. Такой формат очень просто получить с использованием опции $outfmt$. 
	    
	    В скрипте указан порог percent identity в 85\%. Это сделано с целью исключения гомологичных последовательностей, содержащих значительное количество $gap$-ов и/или несовпадений.
	    
	    \item \bf Подготовка последовательностей для Augustus \normalfont
	    
	    Каждая последовательность из output'а blastn, полученная на предыдущем этапе, 
	    проверяется на наличие start, stop кодонов и ORF (скрипты $get\_homologous\_sequences$ и $filter\_seq.py$).
	    
	    Если последовательность не является корректной, то она не входит в training set Augustus'a для поиска генов в Arabidopsis thaliana.
	    
	    Training set сохраняется в формате genbank или gff.
	
	
	\end{itemize}

\end{document}

 